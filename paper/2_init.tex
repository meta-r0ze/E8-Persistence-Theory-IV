\CatchFileBetweenTags{\SpectralIndexVal}{calculations/cosmology.tex}{SpectralIndexVal}
\CatchFileBetweenTags{\SpectralIndexEq}{calculations/cosmology.tex}{SpectralIndexEq}
\CatchFileBetweenTags{\SpectralIndexExperimentalValue}{calculations/cosmology.tex}{SpectralIndexExperimentalValue}
\CatchFileBetweenTags{\SpectralIndexAccText}{calculations/cosmology.tex}{SpectralIndexAccText}

\section{The Primordial Universe: Geometric Initialization and Inflation}
Standard cosmology posits an ad-hoc ``inflation field" to explain the flatness and horizon problems. In our framework, inflation is not driven by a new field, but is the \textbf{geometric phase transition} of the lattice itself as it initializes its causal structure.

\subsection{The Mechanism: Chiral Truncation}
As established in Paper I, the primordial $E_8$ substrate possesses a total symmetric capacity of $N=32$ degrees of freedom, allowing for bidirectional information flow. To establish a causal Arrow of Time, the Persistence Principle forces a \textbf{Chiral Truncation}, reducing the active state space to the chiral sector of $\nu=16$ channels. This transition is inflation: the propagation of the coordination signal that establishes the causal grid of the universe. This geometric mechanism sets the primordial boundary conditions.

\subsection{The Spectral Index (\texorpdfstring{$n_s$}{ns}): Quantization Noise}
The spectral index $n_s$ measures the deviation from perfect scale-invariance ($n_s=1$). A continuous vacuum would be perfectly scale-invariant. However, the $E_8$ lattice is discrete, with its primordial state space defined by $N=2\nu \space (32)$. The projection of this discrete grid onto a continuous manifold is inherently lossy, generating \textbf{Quantization Noise} ($S_Q$). The spectral index is simply the unitary fidelity of the projection minus this noise floor.

\begin{equation}
n_s = 1 - S_Q = \SpectralIndexEq = 1 - \frac{1}{32} = \mathbf{\SpectralIndexVal}
\end{equation}

Experimental value: $\SpectralIndexExperimentalValue$. This geometric prediction of $n_s = \SpectralIndexVal$ \SpectralIndexAccText.  This identifying the ``red tilt" of the primordial spectrum as the digital quantization noise of the underlying 32-channel hardware.

\subsection{Tensor Modes (\texorpdfstring{$r$}{r}): The Gravity Prohibition}
The tensor-to-scalar ratio $r$ measures the amplitude of primordial gravitational waves. In this framework, gravity is not a fundamental force, but an emergent phenomenon caused by the ``load" of massive topological knots on the lattice (as derived in Paper I, System VI). During the inflationary epoch, the universe is a pure scalar field expansion; topological knots (matter) have not yet condensed. With zero structural load on the lattice, the source term for gravitational waves vanishes.
\begin{equation}
r \equiv 0 \quad \text{(No Lattice Load)}
\end{equation}
This theory predicts a null result for all searches for primordial tensor modes. Future experiments like CMB-S4 and LiteBIRD will serve as a hard falsification test for this prediction against standard inflationary models.

\CatchFileBetweenTags{\EtaBaryonVal}{calculations/cosmology.tex}{EtaBaryonVal}
\CatchFileBetweenTags{\EtaBaryonEq}{calculations/cosmology.tex}{EtaBaryonEq}
\CatchFileBetweenTags{\EtaBaryonExperimentalValue}{calculations/cosmology.tex}{EtaBaryonExperimentalValue}
\CatchFileBetweenTags{\EtaBaryonAccText}{calculations/cosmology.tex}{EtaBaryonAccText}

\subsection{The Origin of Matter: Baryogenesis and the Baryon-to-Photon Ratio (\texorpdfstring{$\eta$}{eta})}
Standard cosmology accepts the observed matter-antimatter asymmetry as an initial condition. Our framework derives this ratio, $\eta$, as a direct geometric consequence of the causal initialization sequence.

\paragraph{The Mechanism}
The Chiral Truncation ($N=32 \to \nu=16$) that establishes the Arrow of Time is not perfectly symmetric. As established in Papers I and III, the Chiral Diode possesses an intrinsic geometric phase offset, quantified by the Jarlskog Invariant ($J \approx 3 \times 10^{-5}$). This creates a microscopic bias, $\epsilon$, in favor of matter over antimatter during the truncation event. This initial bias is then diluted by the geometric volume of the newly initialized manifold.

\paragraph{The Derivation}
The final baryon-to-photon ratio is the product of the initial bit bias and the geometric dilution factor, $\zeta$.
\begin{enumerate}
    \item \textbf{Initial Bit Bias ($\epsilon$):} The fractional surplus of matter states is the Jarlskog Invariant distributed over the available chiral channels.
    \begin{equation}
        \epsilon = \frac{J}{\nu} = \frac{3.08 \times 10^{-5}}{16} \approx 1.925 \times 10^{-6}
    \end{equation}
    
    \item \textbf{Geometric Dilution ($\zeta$):} This bias is diluted across the inflationary volume, which we define as the product of the Resonant Expansion ($\pi\Delta$) and the Structural Capacity ($H_{sys} + \chi/D$).
    \begin{equation}
    \begin{split}
        \zeta & = \frac{1}{(\pi \Delta) \cdot (H_{sys} + \chi/D)} \\
        & = \frac{1}{(43\pi) \cdot (23 + 2/4)} \\
        & \approx 3.15 \times 10^{-4}
    \end{split}
    \end{equation}
\end{enumerate}
The final ratio is the product of these two factors:

\begin{equation}
\eta = \epsilon \cdot \zeta = (1.925 \times 10^{-6}) \times (3.15 \times 10^{-4}) \approx \mathbf{\EtaBaryonVal}
\end{equation}

Experimental Value: \EtaBaryonExperimentalValue
This geometric prediction \EtaBaryonAccText


\CatchFileBetweenTags{\zeqVal}{calculations/cosmology.tex}{zeqVal}
\CatchFileBetweenTags{\zeqExperimentalValue}{calculations/cosmology.tex}{zeqExperimentalValue}
\CatchFileBetweenTags{\zeqAccText}{calculations/cosmology.tex}{zeqAccText}

\subsection{Matter-Radiation Equality (\texorpdfstring{$z_{eq}$}{zeq})}
The redshift of matter-radiation equality, $z_{eq}$, marks the epoch where the energy density of matter (structural knots) equals that of radiation (lattice vibrations). In our framework, this is the geometric crossover point where the universe's \textbf{Structural Load} overcomes its \textbf{Resonant Signal}. We derive this scale as the product of the radiation geometry and the matter capacity.
\begin{enumerate}
    \item \textbf{Radiation Scale ($\pi \Delta$):} Radiation modes propagate along the resonant circumference of the lattice, defining the effective "wavelength" of the vacuum energy ($\sim 135.09$).
    \item \textbf{Matter Capacity ($H_{sys} + \chi$):} Matter is constrained by the total degrees of freedom available for storing structural information: the Systemic Channel ($H_{sys}=23$) plus the Topological Boundary ($\chi=2$), for a total of 25 channels.
\end{enumerate}
The equality redshift is the product of these two factors, representing the expansion required to dilute the structural density to match the resonant background.
\begin{equation}
z_{eq} = (\pi \Delta) \times (H_{sys} + \chi) = 135.088 \times 25 = \mathbf{\zeqVal}
\end{equation}
Experimental Value: \zeqExperimentalValue
This geometric prediction \zeqAccText




\section{The Unloaded Universe: The Base Cosmological Rate}
Following inflation, the universe is a hot, radiation-dominated plasma. In this ``unloaded" state, the lattice has low structural complexity, and its dynamics are governed by the pure geometric ratios derived in Paper I.

The expansion rate of this early universe ($H_0^{early}$) represents the ``idle clock speed" of the vacuum hardware. We derive this rate from the same geometric partition that governs the electroweak sector, unifying cosmology with particle physics. The base expansion rate is proportional to the ratio of the total active degrees of freedom (the Systemic Channel, $H_{sys}=23$) to the fundamental resonant depth of the lattice ($\Delta=43$).
\begin{equation}
H_0^{early} \propto \frac{H_{sys}}{\Delta} = \frac{\nu + \sigma + \chi}{\Delta} = \frac{23}{43}
\end{equation}
This geometric ratio, when scaled to cosmological units, corresponds to the Planck 2018 measurement of $H_0 \approx \mathbf{67.4 \pm 0.5}$ km/s/Mpc. This value represents the fundamental, unloaded expansion rate of the vacuum substrate.


\CatchFileBetweenTags{\LoadFactorVal}{calculations/cosmology.tex}{LoadFactorVal}
\CatchFileBetweenTags{\LoadFactorEq}{calculations/cosmology.tex}{LoadFactorEq}

\section{The Loaded Universe: Derivation of the Lattice Load Factor}
The late universe is dominated by matter—heavy topological knots ($\Delta^2, \Delta^3$) that place a significant structural load on the vacuum. To maintain causal coherence under this load, the lattice must increase its effective update frequency. This speed-up is quantified by the \textbf{Lattice Load Factor ($\mathcal{L}$)}. We derive $\mathcal{L}$ from two independent, orthogonal stress terms.

\begin{itemize}
    \item \textbf{Quantization Stress ($S_Q$):} The irreducible noise from the finite bit-depth of the active chiral state space ($\nu=16$). It is the digital overhead of the projection.
    \begin{equation}
        S_Q = \frac{1}{2\nu} = \frac{1}{32} = 0.03125
    \end{equation}
    
    \item \textbf{Topological Stress ($S_T$):} The pressure exerted by the continuous Heegner resonance ($\Delta=43$) against the discrete topological boundary ($\chi=2$). It is the analog overhead of the projection.
    \begin{equation}
        S_T = \frac{\chi}{\Delta} = \frac{2}{43} \approx 0.04651
    \end{equation}
\end{itemize}

These stresses are geometrically orthogonal—one acts on the discrete channel count, the other on the continuous manifold curvature—and therefore sum linearly. The total load factor is the unitary baseline plus these two overheads.
\begin{equation}
\mathcal{L} = 1 + S_Q + S_T = 1 + 0.03125 + 0.04651 = \mathbf{\LoadFactorVal} \\
\end{equation}