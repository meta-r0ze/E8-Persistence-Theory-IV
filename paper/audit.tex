\section{Comparison to the \texorpdfstring{$\Lambda$CDM}{LambdaCDM} Model}
The distinctions between the standard $\Lambda$CDM model and the $E_8$-Persistence Theory are structural. Where $\Lambda$CDM relies on fitted parameters to describe observations, our framework derives these quantities from geometric first principles and treats the vacuum not as a passive background, but as a dynamic, finite-capacity substrate.

\begin{table}[h!]
\centering
\caption{$\Lambda$CDM Cosmological Parameters}
\label{tab:lcdm_parameters}
\begin{tabular}{l l}
\hline
\textbf{Parameter} & \textbf{$\Lambda$CDM (Descriptive)} \\
\hline
Vacuum Energy ($\rho_{\Lambda}$) & Fine-tuned input \\
Hubble Constant ($H_0$) & Tension (67 vs. 73) \\
Dark Matter ($\Omega_{DM}$) & Unknown Particle \\
Dark Energy ($w$) & Assumed Constant ($-1$) \\
Neutrino Mass ($\Sigma m_\nu$) & Tension \\
\hline
\end{tabular}
\end{table}


\section{Structural Audit: A Call for Verification}
The ``Hubble Tension" is currently framed as a conflict between datasets. We propose it is a conflict of models. We invite the cosmology community to test the Lattice Load ($\mathcal{L}$) model directly with existing data.

\paragraph{The ``Loaded" $\Lambda$CDM Fit}
A direct test of our central hypothesis can be performed by re-analyzing the Planck CMB data chains.
\begin{itemize}
    \item \textbf{Task:} Introduce the Lattice Load Factor ($\mathcal{L} \approx 1.078$) as a discrete, hard-coded step-function for the late-universe expansion rate $H(z)$ for $z < z_{eq}$.
    \item \textbf{Expected Result:} This ``Loaded" model should resolve the tension with SH0ES data more effectively (i.e., with higher Bayesian evidence) than continuous Early Dark Energy models.
\end{itemize}

\paragraph{The Neutrino Mass Sum Constraint}
Our framework provides a hard prediction for the gravitational mass of neutrinos, which can be used as a fixed prior in Large Scale Structure (LSS) analyses.
\begin{itemize}
    \item \textbf{Task:} Analyze LSS data (e.g., from DESI) fixing the gravitational neutrino mass sum to the geometric phonon floor, $\Sigma m_\nu \approx 9.4$ meV.
    \item \textbf{Expected Result:} This should provide an improved fit to structure growth parameters ($f\sigma_8$) compared to a model where $\Sigma m_\nu$ is a free parameter.
    \item \textbf{Falsification Criterion:} A definitive measurement from LSS data showing $\Sigma m_\nu > 0.05$ eV would strongly challenge the geometric phonon model.
\end{itemize}
