\section{Strong-Field Gravity: Black Holes and Information}
The emergent gravity model, derived from the lattice's information-theoretic properties in Paper I, makes unique and falsifiable predictions in strong-field regimes where General Relativity is stressed. We now explore the consequences of \textbf{Lattice Saturation}, where high information density pushes the vacuum substrate to its computational limit.

\paragraph{The Refractive Index Limit}
As established in Paper I, gravity is not a fundamental force but a manifestation of the vacuum's variable refractive index $n(r)$, which is a function of the local information density $K(r)$ relative to the lattice's chiral channel capacity $\nu=16$.
\begin{equation}
n(r) = \frac{\nu}{B_{res}(r)} = \frac{\nu}{\nu - K(r)}
\end{equation}
In the weak-field limit, this recovers the Schwarzschild metric. However, as $K(r) \to \nu$, this non-linear dependence leads to deviations from standard GR.

\subsection{Pre-Saturation: A Testable Violation of Birkhoff's Theorem}
Standard GR, via Birkhoff's Theorem, asserts that the external gravitational field of a spherical object depends only on its total mass, not its internal density distribution. Our framework predicts a subtle violation of this principle.

The information density $K(r)$ of a diffuse gas cloud is distributed linearly across many lattice nodes. A dense neutron star, however, concentrates this load, triggering the non-linear saturation penalty in the refractive index.
\begin{itemize}
    \item \textbf{Prediction:} A dense neutron star will produce a slightly stronger time dilation field (a higher effective $n(r)$) than a diffuse gas cloud of the exact same total mass. This structure-dependent effect falsifies the strict equivalence principle.
    \item \textbf{Verification:} This deviation, estimated to be at the $\sim 0.1-0.3\%$ level for typical neutron stars, should be detectable by next-generation Pulsar Timing Arrays.
\end{itemize}

\subsection{Total Saturation: The Black Hole Event Horizon}
A black hole, in this framework, is not a singularity of infinite density, but a region of total \textbf{Bandwidth Saturation}.
\begin{itemize}
    \item \textbf{The Limit:} As infalling matter concentrates, the local information density approaches the channel capacity ($K(r) \to \nu$). The residual capacity $B_{res}(r) \to 0$, causing the refractive index to diverge ($n(r) \to \infty$).
    \item \textbf{The Horizon:} This divergence defines the event horizon. Since the local speed of causality scales as $c' = c/n$, the ``clock rate" at the horizon freezes to zero relative to an external observer. The lattice nodes at the horizon are fully occupied maintaining their existing state and possess zero bandwidth to process new infalling information.
\end{itemize}

\subsection{Resolution of the Black Hole Information Paradox}
The information paradox arises from the apparent loss of unitary information during Hawking evaporation. The lattice saturation model resolves this by redefining the nature of the event horizon and its radiation.
\begin{itemize}
    \item \textbf{Information Buffering, Not Erasure:} Information is not destroyed at a singularity. It is \textbf{buffered} at the saturation boundary. The event horizon acts as a ``write-protect" surface where the lattice density is maximized. The interior is a region of uniform saturation where the causal update rate is frozen.
    \item \textbf{Hawking Radiation as Information Reflection:} Because a saturated node has zero residual capacity, it cannot ingest new bit strings. Infalling information and quantum fluctuations are \textbf{rejected} at this boundary. This forced reflection of information off the ``computational wall" of the horizon is the physical mechanism for Hawking radiation.
\end{itemize}
In this view, the black hole information paradox is an artifact of assuming a continuous spacetime. In a finite-capacity substrate, information cannot be destroyed, only buffered or reflected.

\subsection{Black Hole Thermodynamics: The Geometric Origin of Entropy and Temperature}
The information-theoretic nature of the event horizon allows for a direct derivation of the geometric constants in the laws of black hole thermodynamics, linking them to the fundamental invariants of the lattice.

\paragraph{The Bekenstein-Hawking Entropy}
The entropy of a black hole is proportional to its horizon area, $A$, but contains a famous factor of $1/4$: $S = A / (4\ell_p^2)$. In our framework, this factor is not arbitrary but is a direct consequence of the holographic principle in a D-dimensional spacetime. Information about the black hole's interior is encoded on the surface of the event horizon and must be projected across the full dimensionality of the manifold.
\begin{equation}
\begin{split}
    S_{BH} &= \frac{\text{Horizon Area}}{\text{Dimensional Projection} \times \text{Planck Area}} \\
    &= \frac{A}{D \cdot \ell_p^2} = \frac{A}{4\ell_p^2}
\end{split}
\end{equation}
The factor of $1/4$ is identified as the inverse of the manifold rank ($D=4$).

\paragraph{The Hawking Temperature}
The temperature of a black hole is inversely proportional to its mass, $T_H \propto 1/M$, but contains a geometric factor of $8\pi$. We derive this factor as the thermodynamic surface area of a topological defect within a D-dimensional manifold. It is the product of the topological boundary's closure, the manifold's dimensionality, and the circular interface constant.
\begin{equation}
\text{Surface Factor} = \chi \cdot D \cdot \pi = 2 \cdot 4 \cdot \pi = 8\pi
\end{equation}
This demonstrates that the laws of black hole thermodynamics are not separate from the Standard Model, but are emergent consequences of the same underlying geometric invariants ($\chi=2, D=4$) that define the structure of particles and forces.

% --- END OF NEW LATEX SUBSECTION ---