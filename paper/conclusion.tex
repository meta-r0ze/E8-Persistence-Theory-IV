\section{Conclusion}
By extending the $E_8$-Persistence Theory to the macroscopic scale, we have resolved the primary tensions of modern cosmology without invoking new fields or free parameters. The "anomalies" of the $\Lambda$CDM model are shown to be the predictable signatures of a finite-capacity universe operating under structural load.

This work presents a unified cosmological model, deriving the parameters of inflation and the dynamics of the late universe from a single set of geometric principles. We have demonstrated that the Hubble and $S_8$ tensions are dual manifestations of Systemic Latency, and that the "dark" sector is the geometric weight and entropic exhaust of the vacuum substrate.

This framework makes the following hard, falsifiable predictions for upcoming experiments:
\begin{itemize}
    \item The Hubble constant will converge to $\mathbf{H_0 = 72.64 \pm 0.5}$ \textbf{km/s/Mpc}.
    \item The cosmological sum of neutrino masses will be measured by Euclid as $\mathbf{\Sigma m_\nu < 0.03}$ \textbf{eV}.
    \item \textbf{All particle-based Dark Matter searches will yield null results.} Our model identifies Dark Matter as the geometric mass of the lattice itself, not a particle. While this explains its abundance and gravitational effects, it also predicts that Dark Matter cannot be directly detected via particle scattering, nor can its fine-grained distribution be mapped by any means other than its gravitational influence.
    \item Future CMB surveys (e.g., CMB-S4) will confirm the absence of primordial tensor modes ($\mathbf{r=0}$).
\end{itemize}
The universe described here is not a fine-tuned stage for arbitrary fields, but a discrete, efficient, and inevitable structure governed by the geometry of persistence.