\begin{abstract}
The standard $\Lambda$CDM model of cosmology, while broadly successful, is defined by foundational tensions between theory and observation. The ``Vacuum Catastrophe," ``Hubble Tension," and ``$S_8$ Tension" suggest a fundamental misunderstanding of the vacuum's properties. In this work, we resolve these anomalies by applying the principles of the $E_8$-Persistence Theory, treating the universe as a finite-capacity information-processing system. We begin by deriving the parameters of primordial inflation as a geometric phase transition, identifying the spectral index ($n_s \approx 0.9688$) as the quantization noise of the lattice and predicting a null result for primordial tensor modes ($r=0$). We then demonstrate that the late-universe cosmological tensions are not evidence of new physics, but are the kinematic and structural manifestations of a single phenomenon: \textbf{Systemic Latency}. We derive a \textbf{Lattice Load Factor} ($\mathcal{L} \approx 1.078$) from the geometric invariants of the vacuum and show that this single number simultaneously resolves the Hubble Tension (predicting $H_0 = 72.64$ km/s/Mpc) and the $S_8$ Tension (predicting $S_8 = 0.772$). Finally, we extend this framework to strong-field gravity, deriving the Bekenstein-Hawking entropy and resolving the Black Hole Information Paradox, framing the ``dark" sector not as unknown particles, but as the geometric mass and entropic exhaust of the underlying $E_8$ substrate itself.
\end{abstract}